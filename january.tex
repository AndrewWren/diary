\chapter{January}

\section{The first day of the year\diarytime{1st January 2015}}
\newrefsection
\diarytime{17:00}
I spent the first day of the year setting up a good environment for a diary, or 
journal as I prefer to call it.
I want to have something similar to a research diary, in which I can document 
my progress in more detail.
This should make it easier for me when it comes to the weekly scrums and the 
monthly and yearly metas.

It is based on Tufte-LaTeX classes \diarycite{tufte-latex}.
It provides three new commands \verb'\diarytime', \verb'\diarycite' and 
\verb'\references'.
The first command, \verb'\diarytime', sets a new date in the margin.
The second command, \verb'\diarycite', prints the full bibliography entry for 
a reference in the margin.
The third command, \verb'\references', prints the list of references for the 
current section.
For this command to work, the \verb'refsection' environment (or the 
\verb'\newrefsection' command) from Biblatex \diarycite{biblatex} must be used.

\subsection{Picking up the slack}
\diarytime{19:00}
I also did some committing into the repository, trying to sort out what I was 
doing before Christmas.
I committed the start of the crypto and meta parts\diarytime{23:30} of my 
reflections.

\references


\section{The second day\diarytime{2nd January 2015}}
\diarytime{15:20}
Today I did some other stuff, stuff which doesn't require any of the fancy 
commands.


\section{Another command to the diary\diarytime{10th January 2015}}
\diarytime{22:00}
I added the command \verb'\topic' to the journal template---or \topic{diary 
class}.
This command places the topic name in the margin and adds an entry in the 
index.
